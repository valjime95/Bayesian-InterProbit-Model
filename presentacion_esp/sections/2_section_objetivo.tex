\setwatermark{\fontsize{74pt}{74pt}\selectfont{\textcolor{mediumelectricblue}{Bayesiana}}}
\begin{frame}{Objetivo}
\textcolor{black}{
El objetivo principal fue aprender como se lleva a cabo la \textbf{inferencia Bayesiana}. \\
Para esto se estudió el modelo de \textit{Yang  y Allenby}\cite{yang2003_interdependent}, en donde se presenta un modelo que hace inferencia acerca de una \alert{preferencia binaria} no sólo con las covariables $X$, si no también incluyendo un segundo término de error al modelo Probit en el cual se toma en cuenta la \textbf{interdependencia} que pudiera exisitir entre los individuos de estudio.}
\end{frame}