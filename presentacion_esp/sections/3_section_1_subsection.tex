\subsection{Construcción}

\setwatermark{\fontsize{10pt}{10pt}\selectfont{ }}

{
\usebackgroundtemplate{\includegraphics[width=\paperwidth]{img/inter.eps}}%
\begin{frame}

\end{frame}
}

\setwatermark{\fontsize{30pt}{30pt}\selectfont{ }}
\begin{frame}{Modelo Probit}
\framesubtitle{aumentado de variables latentes}

\begin{columns}
\column{.5\textwidth}

\vspace*{.5 cm}
\hspace*{.2cm}
\hbox{%
\Huge
 \transparent{0.5}%
\textcolor{applegreen}{Decisiones}
 }

\vspace*{1.0 cm}
\hspace*{.2cm}
 \hbox{%
\Large
 % \transparent{0.9}%
\textcolor{beaublue}{Maximización de la}
 }

\vspace*{.3 cm}
\hspace*{.5cm}
 \hbox{%
\Huge
 % \transparent{0.9}%
\textcolor{beaublue}{utilidad}
 }

\column{.5\textwidth}

\[ y_{i} = \begin{cases*}
  1 & si se elige A \\
  0 & si se elige B\\ \end{cases*}. \]


\[
\begin{split}
\mathbb{P} (y_{i}=1) &= \mathbb{P} (U_{i,A} >U_{i,B}) \\
&=  \mathbb{P} (U_{i,A} - U_{i,B} > 0) \\
&=  \mathbb{P} (z_i > 0).
\end{split}
\]

  
\end{columns}

\vspace*{1 cm}
\hspace*{.6cm}
\hbox{%
\fontsize{20}{20}\selectfont
 \transparent{0.6}%
\textcolor{darkpastelpurple}{Probit de variables latentes}
 }

 \[ z_i = x_i\beta + \varepsilon_i \qquad  \varepsilon_i \sim N(0,1) \]

\end{frame}

% ---------------------- PROBIT -------------------------------------
\setwatermark{\fontsize{90pt}{90pt}\selectfont{\textcolor{indigo(web)}{Probit}}}
\begin{frame}{Modelo Probit}
\framesubtitle{Componentes}

\[ z_i = x_i\beta + \varepsilon_i \qquad  \varepsilon_i \sim N(0,1) \]

\textcolor{black}{
donde:\\
$z_i$ = Preferencia latente del individuo $i$.\\
$x_i$ = vector de covariables.\\
$\beta$ = vector de coeficientes asociados a $x_i$.\\
$\varepsilon_i$ = coeficiente de error}

% $z_i$ = Preferencia latente del individuo $i$ de elegir A sobre B.\\
% $z_i$ = vector de covariables que captura las cualidades del individuo $i$ así como la captura de la diferencia entre las cualidades de las alternativas.\\
% $\beta$ = vector de coeficientes asociados a $x_i$.\\
% $\varepsilon_i$ = término de error que refleja los factores no observados.}

\end{frame}

% --------------------- PROBLEMATICA --------------------------------------
\setwatermark{\fontsize{40pt}{40pt}\selectfont{ }}
\begin{frame}{Modelo Probit}
\framesubtitle{Problemática}

Al tener a $\varepsilon_i$ \alert{independiente} a través de los individuos, se refleja la \alert{ausencia de posibles efectos de dependencia} entre las preferencias de los individuos.

\vspace*{1.0 cm}
 \hbox{%
\fontsize{40}{48}\selectfont
 \transparent{0.3}%
\textcolor{mint}{James LeSage\textsubscript{\cite{lesage2000}}}
 }\vbox{%
 \fontsize{8}{8}\selectfont
 \transparent{0.3}%
\textcolor{mint}{\cite{lesage2000}}}

consideró introducir una matriz de covarianzas de $\varepsilon$ a través de los individuos, mediante un parámetro \alert{autorregresivo} $\theta$,

\[ \theta = \rho W\theta + u \qquad  u \sim N(0,\sigma_{u}^2 I_m). \]

\end{frame}

% ------------------------ THETHA 1 ------------------------------------
\subsubsection{Parámetro autoregresivo $\theta$}
\setwatermark{\fontsize{40pt}{40pt}\selectfont{ }}
\begin{frame}{Modelo InterProbit}
\framesubtitle{Efecto regional}

\begin{variableblock}{$\theta$: parámetro autoregresivo}{bg=white,fg=normal}{bg=fondo,fg=destacado}
  Representa todas las \textcolor{mediumslateblue}{dependencias no observadas} entre los individuos. 
\end{variableblock}

\vspace*{.5 cm}
 \hbox{%
\fontsize{20}{20}\selectfont
 \transparent{0.6}%
\textcolor{iris}{Es autorregresivo}
 }

porque describe las variaciones que exhiben las preferencias latentes medidas en diferentes lugares.

\vspace*{.5 cm}
 \hbox{%
\fontsize{20}{20}\selectfont
 \transparent{0.3}%
\textcolor{mediumslateblue}{Entonces ...}
 }
los aspectos de estas dependencias pueden ser parecidas o inclusive iguales a las de individuos cercanos.
\end{frame}
% ------------------------ THETA 2------------------------------------

\setwatermark{\fontsize{50pt}{50pt}\selectfont{\textcolor{frenchblue}{dependencia}}}
\begin{frame}{Modelo InterProbit}
\framesubtitle{Efecto regional}

\begin{variableblock}{$\theta$: parámetro autoregresivo}{bg=white,fg=normal}{bg=beaublue,fg=black}

\begin{align}
& \theta = \rho W\theta + u \qquad  u \sim N(0,\sigma_{u}^2I_m ) \nonumber \\ 
& \theta = \left(I_m -\rho W \right)^{-1}u \nonumber
\end{align}

donde:\\
$\rho$: grado de dependencia entre los individuos \\
$W$ : matriz de pesos $(m \times m)$\\
$u$ : vector de errores iid de tamaño $(m \times 1)$\\
$\sigma^2_u$ : incertidumbre de la estructura espacial
\end{variableblock}

\end{frame}
% ------------------------ INTERPROBIT ------------------------------------
\setwatermark{\fontsize{50pt}{50pt}\selectfont{ }}
\begin{frame}{Modelo InterProbit}
% \framesubtitle{Efecto regional}

\begin{variableblock}{Modelo Probit de preferencias espaciales y demográficas}{bg=white,fg=normal}{bg=probit_claro,fg=black}

\begin{align}
& Z = X\beta +\varepsilon + \theta,  \nonumber \\
& \theta = \rho W \theta + u,  \nonumber \\
& \varepsilon \sim N(0,I_m), \nonumber \\
& u \sim N(0,\sigma^{2}I_m). \nonumber
\end{align}
\end{variableblock}

Al haber aumentado un término de error, $\theta$,  que determine que las covarianzas sean distintas de cero en las preferencias latentes, y que además no se encuentre en la verosimilitud, se toman en cuenta los efectos de las covarianzas distintas de cero  y se simplifica la evaluación de la función de verosimilitud

\[
Z \sim N \left(X\beta, I_m + \sigma^2 \left[\left(I_m-\rho W\right) \left(I_m-\rho W\right)' \right]^{-1} \right). \]


\end{frame}

% ---------------------------- RHO 1---------------------------
\setwatermark{\fontsize{90pt}{90pt}\selectfont{\textcolor{frenchblue}{Probit}}}
\begin{frame}[t]{Modelo InterProbit}
\framesubtitle{Componentes de $\theta$: $\rho$}

% \begin{variableblock}{$\rho$}{bg=white,fg=normal}{bg=mint_claro,fg=black}
\begin{columns}
\column{.5\textwidth}

\vspace*{.5 cm}
\begin{figure}[h]
\includegraphics[scale = .25]{./img/rho.png}
\end{figure}

\column{.5\textwidth}

\vspace*{.5 cm}
\textcolor{black}{
Es la fuerza de dependencia espacial y demográfica entre los individuos.}\\
\end{columns}
\vspace*{.4 cm}
% \end{variableblock}

\begin{itemize}
\item \textcolor{black}{Dada $W$, $\rho$ representa el impacto de primer orden.}
\item \textcolor{black}{$\frac{1}{\lambda_{\min}} < \rho < \frac{1}{\lambda_{\max}}.$}
\item \textcolor{black}{Si $\rho > 0 $ entonces existe una dependencia positiva entre los individuos.}
\end{itemize}

\end{frame}

% ----------------------------- MATRIZ W 2 -----------------------------
\setwatermark{\fontsize{90pt}{90pt}\selectfont{\textcolor{mint}{Inter}}}
\begin{frame}{Modelo InterProbit}
\framesubtitle{Componentes de $\theta$: $W$}

\begin{figure}[H]\def\b{\color{mint}}
\centering
\begin{minipage}[b]{.45\textwidth}
\begin{tikzpicture}[mystyle/.style={draw,shape=circle,fill=mint_claro, inner sep=0pt, minimum size=4pt, label={[anchor=center, label distance=2mm](90+360/\ngon*(####1-1)):####1}}]
\def\ngon{5}
\node[draw, regular polygon,regular polygon sides=\ngon,minimum size=3cm] (p) {};
\foreach\x in {1,...,\ngon}{
    \node[mystyle=\x] (p\x) at (p.corner \x){};
}
\end{tikzpicture}
\caption{Gráfico de conexión circular del modelo.}\label{first}
\end{minipage}\hfill
\begin{minipage}[b]{.45\textwidth}
%\begin{figure}
\centering
\small
\label{M}
\(
W= 
\begin{pmatrix}
0 & \b\text{.5} & 0 & 0 & \b\text{.5} \\
\b\text{.5} & 0 & \b\text{.5} & 0 & 0 \\
0 & \b\text{.5} & 0 & \b\text{.5} & 0 \\
0 & 0 & \b\text{.5} & 0 & \b\text{.5} \\
\b\text{.5} & 0 & 0 & \b\text{.5} & 0 \\
\end{pmatrix}
\)
\caption{Estructura de la matriz $W$ (normalizada) de 5 individuos.}\label{second}
%\end{figure}
\end{minipage}
\end{figure}

\end{frame}

% ----------------------- MATRIZ W 1 -------------------------------------
\setwatermark{\fontsize{60pt}{60pt}\selectfont{ }}
\begin{frame}{Modelo InterProbit}
\framesubtitle{Componentes de $\theta$: $W$}

\begin{variableblock}{$W$: matriz de pesos espaciales y demográficos}{bg=white,fg=normal}{bg=mint_claro,fg=black}

Sea $W_{\left(n \times n\right)}$ una matriz, se dice que $W_{\left(n \times n\right)}$ es la \textit{matriz de pesos} si el elemento $w_{i,j}$, $i,j = 1,\dots,n$, de la matriz describe la cercanía entre el elemento $i$ con el $j$ en términos de una medida de distancia, esto es:

\[
w_{i,j} = 
  \begin{cases*}
  1 & \text{el elemento $i$ es vecino contiguo de $j$} \\
  0 & \text{el elemento $i$ no es vecino contiguo de $j$} \\
  \end{cases*}.
\]

Los elementos que se ven como vecinos de un elemento dado interactúan con ese elemento de una manera significativa.
Como al elemento $i$ no se le considera como su propio vecino entonces $w_{i,i} = 0$.
\end{variableblock}

\end{frame}

% ----------------------- MATRIZ W 3 -------------------------------------
\setwatermark{\fontsize{60pt}{60pt}\selectfont{ }}
\begin{frame}{Modelo InterProbit}
\framesubtitle{Componentes de $\theta$: $W$ con múltiples covariables}

Yang y Allenby\cite{yang2003_interdependent} especifican un matriz autorregresiva $W$ como una mezcla finita de matrices de coeficientes, cada una relacionada con una covariable específica:

\begin{gather}
  W = \sum_{k=1}^{K} \phi_k W_k \\
  \sum_{k=1}^{K} \phi_k = 1,
\end{gather}

\end{frame}

